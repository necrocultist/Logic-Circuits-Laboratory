\documentclass[pdftex,12pt,a4paper]{article}
\usepackage{breakcites}
\usepackage{indentfirst}
\usepackage{pgfgantt}
\usepackage{pdflscape}
\usepackage{float}
\usepackage{epsfig}
\usepackage{epstopdf}
\usepackage[cmex10]{amsmath}
\usepackage{stfloats}
\usepackage{multirow}
\usepackage{mathtools}
\usepackage{karnaugh-map}
\usepackage{subfiles}
\usepackage{environ}
\usepackage{amssymb}
\usepackage{amsthm}
\usepackage{amssymb}
\theoremstyle{plain}
\usetikzlibrary{intersections}
\newlength{\crossing}
\makeatletter
\makeatother

\title{Lab - HW3}
\author{Kaan Karataş}
\date{April 2023}

\newcommand{\circuit}[1]{
    \begin{figure}[H]
    	\centering
    	\includegraphics[width=1\textwidth]{circuits/#1_circuit.png}
    	\caption{#1 Circuit}
    	\label{fig7}
    \end{figure}
    \vspace{1cm}
}

\thispagestyle{empty}
\begin{document}

\subfile{cover.tex}

\thispagestyle{empty}


\setcounter{tocdepth}{4}
\tableofcontents
\clearpage
\setcounter{page}{1}
\setcounter{subsubsection}{0}

\section{INTRODUCTION}
Throughout this experiment, the C.A.D.E.T. was tested while a half adder, full adder and adder subtractor circuits were designed. According to the circuits, truth tables were produced, and the outcomes were compared to the C.A.D.E.T. unit. The experiments’ primary goal is to create adders and later use them to design a 4 bit adder subtractor circuit.

\section{MATERIALS AND METHODS}
Tools used on this experiment:
\begin{itemize}
    \item C.A.D.E.T
    \item 7400 series ICs
    \begin{itemize}
        \item 74xx08 - Quadruple 2-input Positive AND Gates
        \item 74xx32 - Quadruple 2-input Positive OR Gates
        \item 74xx83 - 4-bit Binary Full Adder
        \item 74xx86 - Quadruple 2-input Positive Exclusive Or (XOR) Gates
    \end{itemize}
\end{itemize}
\newpage

\section{PRELIMINERY}
    \begin{itemize}
        \item When performing binary addition for unsigned numbers, each bit position is added up, and any carries from lower positions are propagated to higher positions. The outcome is the two numbers added together. We first represent the two numbers in two's complement form and then add them using binary addition for signed addition. The most important bit, which represents the sign of the outcome, is then discarded if it contains any carry. 
        \item When performing binary subtraction for unsigned numbers, each position of a bit is subtracted, and any borrow from a lower position is propagated to the following higher position. The outcome is the difference between the two numbers. We first represent the two numbers in two's complement form and then subtract them using binary subtraction to perform signed subtraction. After that, we throw away any borrows from the most important bit, which represents the outcome's sign. 
        \item When the result of an addition or subtraction operation exceeds the maximum value that can be represented by the number of bits used for the calculation, the terms "carry" and "borrow" are used. A borrow happens when a subtraction operation needs an extra bit to represent the difference, whereas a carry happens when an addition operation needs an extra bit to represent the sum. These ideas are employed to assess whether a result is accurate or not.
        \item When the outcome of an arithmetic operation exceeds the highest value that the number of bits employed for the calculation can represent, it is said to have overflowed. Both addition and subtraction operations can have it. When an overflow happens, the calculation should be redone using more bits because the result is invalid. 
    \end{itemize}

\newpage
\section{EXPERIMENT}
    \subsection{Part 1}
        A half adder is implemented using 1 AND and 1 XOR gate with 2 inputs labeled A and B and 2 outputs representing Sum and Carry.
        \circuit{Half Adder}
        
        \begin{center}
            \begin{tabular}{c c | c c}
                A & B & Carry & Sum \\
                \hline 
                0 & 0 & 0 & 0 \\
                0 & 1 & 0 & 1 \\
                1 & 0 & 0 & 1 \\
                1 & 1 & 1 & 0 \\
            \end{tabular}
        \end{center}
        {\centering Truth Table of Half Adder\\}

    \newpage
    \subsection{Part 2}
        A full adder is implemented by adding an additional $C_{in}$ input.
        \circuit{Full Adder}

        \begin{center}
            \begin{tabular}{c c c | c c}
                $C_{in}$ & $A$ & $B$ & $Sum$ & $C_{out}$ \\
                \hline 
                0 & 0 & 0 & 0 & 0\\
                0 & 0 & 1 & 1 & 0\\
                0 & 1 & 0 & 1 & 0\\
                0 & 1 & 1 & 0 & 1\\
                1 & 0 & 0 & 1 & 0\\
                1 & 0 & 1 & 0 & 1\\
                1 & 1 & 0 & 0 & 1\\
                1 & 1 & 1 & 1 & 1\\
            \end{tabular}
        \end{center}
        {\centering Truth Table of Full Adder\\}
        
    \newpage
    \subsection{Part 3}
    \circuit{4-bit Full Adder}
    \circuit{Adder Subtractor}
        \begin{center}
            \begin{tabular}{c c | c c c}
                A & B & Carry & Binary & Decimal \\
                \hline 
                0101 & 0111 & 0 & 1100 & 12 \\
                1101 & 1001 & 1 & 0110 & 22 (extra bit)\\
                1111 & 1111 & 1 & 1110 & 30 (extra bit)\\
                0110 & 1101 & 1 & 0011 & 19 (extra bit)\\
            \end{tabular}
            \vspace{0.3cm}
        \end{center}
        {\centering Unsigned A + B\\}

        \begin{center}
            \vspace{0.3cm}
            \begin{tabular}{c c | c c c}
                A & B & Overflow & Binary & Decimal \\
                \hline 
                0101 & 0111 & 1 & 1100 & -3 (can not be represented) \\
                1101 & 1001 & 1 & 0110 & 10 (can not be represented)\\
                1111 & 1111 & 0 & 1110 & -2\\
                0110 & 1101 & 0 & 0011 & 3\\
            \end{tabular}
            \vspace{0.3cm}
        \end{center}
        {\centering Signed A + B\\}

        \begin{center}
            \vspace{0.3cm}
            \begin{tabular}{c c | c c c}
                A & B & Barrow & Binary & Decimal \\
                \hline 
                0101 & 0111 & 1 & 1110 & 13 (can not be represented) \\
                1101 & 1001 & 0 & 0100 & 4\\
                1111 & 1111 & 0 & 0000 & 0\\
                0110 & 1101 & 1 & 1001 & 9 (can not be represented)\\
            \end{tabular}
            \vspace{0.3cm}
        \end{center}
        {\centering Unsigned A - B\\}

        \begin{center}
            \vspace{0.3cm}
            \begin{tabular}{c c | c c c}
                A & B & Overflow & Binary & Decimal \\
                \hline 
                0101 & 0111 & 0 & 1110 & -2\\
                1101 & 1001 & 0 & 0100 & 4\\
                1111 & 1111 & 0 & 0000 & 0\\
                0110 & 1101 & 1 & 1001 & 9 (can not be represented)\\
            \end{tabular}
            \vspace{0.3cm}
        \end{center}
        {\centering Signed A - B\\}
        
\newpage
\section{RESULTS}
Every circuit design is implemented using Logisim. The intended outputs are obtained when the circuits are configured with all of the provided functionalities. A half adder, full adder and adder subtractor circuits are implemented. The circuits' outcomes were consistent with their truth tables.

\section{DISCUSSION}
In the experiment, we learnt to implement a half adder, and use the adder to make a full adder by adding additional carry input. After that, by using the full adder we designed a adder subtractor circuit and tested it with different outputs.
\section{CONCLUSION}
We completed the experiment as it was not a hard one. However, due to an issue that could not get identified by instructors, adder subtractor ciruit's second bit did not work as expected. Despite testing every possibility that could cause the issue including wires, gates, LEDs etc. nothing resolved the problem despite the fact that the connections were certainly correct as stated by many instructors. The most likely and only left untested issue was probably the  C.A.D.E.T. device.
\end{document}